\documentclass[10pt,landscape]{article}
\usepackage{multicol}
\usepackage{calc}
\usepackage{ifthen}
\usepackage[landscape]{geometry}
\usepackage{amsmath,amsthm,amsfonts,amssymb}
\usepackage{color,graphicx,overpic}
\usepackage{hyperref}
\usepackage{enumitem}
\usepackage{tikz}
\usepackage{tikz-cd}
\usetikzlibrary{arrows,positioning}

% Preamble imports
\newcommand{\R}{\mathbb{R}}
\newcommand{\N}{\mathbb{N}}
\newcommand{\Z}{\mathbb{Z}}
\newcommand{\C}{\mathbb{C}}
\newcommand{\Q}{\mathbb{Q}}
\newcommand{\dist}{\operatorname{dist}}
\newcommand{\ovl}[1]{\overline{#1}}

% Page setup
\geometry{top=.4in,left=.4in,right=.4in,bottom=.4in}
\pagestyle{empty}

% Redefine section commands to use less space
\makeatletter
\renewcommand{\section}{\@startsection{section}{1}{0mm}%
                                {-1ex plus -.5ex minus -.2ex}%
                                {0.5ex plus .2ex}%x
                                {\normalfont\large\bfseries}}
\renewcommand{\subsection}{\@startsection{subsection}{2}{0mm}%
                                {-1explus -.5ex minus -.2ex}%
                                {0.5ex plus .2ex}%
                                {\normalfont\normalsize\bfseries}}
\renewcommand{\subsubsection}{\@startsection{subsubsection}{3}{0mm}%
                                {-1ex plus -.5ex minus -.2ex}%
                                {1ex plus .2ex}%
                                {\normalfont\small\bfseries}}
\makeatother

% Don't print section numbers
\setcounter{secnumdepth}{0}

\setlength{\parindent}{0pt}
\setlength{\parskip}{0pt plus 0.5ex}

% Compact lists
\setlist{nosep, leftmargin=*}

\begin{document}
\raggedright
\footnotesize
\begin{multicols}{3}

% multicol parameters
\setlength{\premulticols}{1pt}
\setlength{\postmulticols}{1pt}
\setlength{\multicolsep}{1pt}
\setlength{\columnsep}{2pt}

\begin{center}
     \Large{\underline{MATH 154 Midterm 1 Sheet}} \\
     \small{Kyle Trinh, FA25}
\end{center}

\section{Ch 1: Introduction}

\textbf{Graph:} $G = (V, E)$ where:
\begin{itemize}
\item $V$ = set of vertices
\item $E$ = set of edges (unordered pairs from $V$)
\end{itemize}

\textbf{Adjacent Vertices:} Connected by an edge.

\textbf{Incident:} Vertex $v$ is incident on edge $e$ if $v$ is an endpoint of $e$.

\textbf{Neighborhood:}
\begin{itemize}
\item Open: $N(v)$ = adjacent vertices (not including $v$)
\item Closed: $\ovl{N(v)} = N(v) \cup \{v\}$
\end{itemize}

\textbf{Subgraph:} $H$ is subgraph of $G$ if obtained by taking some vertices and edges of $G$.

\textbf{Induced Subgraph:} Takes vertices and \textit{all} edges between them.

\textbf{Pseudograph:} Graph allowing self-loops.

\textbf{Multigraph:} Graph allowing multiple edges between same vertices.

\textbf{Directed Graph (Digraph):} Edges have direction.

\subsection{Degree}

\textbf{Degree $\deg(v)$:} Number of edges incident on $v$.
\begin{itemize}
\item In multigraph: count each edge
\item Self-loop counts twice
\end{itemize}

\textbf{$d$-regular:} All vertices have degree $d$.

\textbf{Minimum Degree:} $\delta(G) = \min_{v \in V} \deg(v)$
\subsection{Handshake Lemma}

\textbf{Theorem:} For finite graph $G = (V, E)$:
$$\sum_{v \in V} \deg(v) = 2|E|$$

\textit{Proof:} Each edge counted twice when summing degrees (once per endpoint).

\textbf{Corollary 1:} Number of odd-degree vertices is even.

\textbf{Corollary 2:} Cannot have degree sequence with exactly one odd-degree vertex.

\textbf{Complete Graph $K_n$:} Every pair of vertices connected.
\begin{itemize}
\item $(n-1)$-regular
\item $|E| = \frac{n(n-1)}{2} = \binom{n}{2}$
\end{itemize}

\textbf{Cycle $C_n$:} Circular arrangement of $n$ vertices.
\begin{itemize}
\item $|V| = n$, $|E| = n$
\item 2-regular
\item Bipartite iff $n$ even
\end{itemize}

\textbf{Path $P_n$:} Linear arrangement of $n$ vertices.
\begin{itemize}
\item $|V| = n$, $|E| = n-1$
\item Two vertices of degree 1, rest degree 2
\end{itemize}

\textbf{Bipartite Graph:} $V = L \cup R$ where $L \cap R = \varnothing$, all edges connect $L$ to $R$.

\textbf{Theorem:} Let $G = H_1 \sqcup H_2$ with only one shared edge. Then $G$ is bipartite iff $H_1$ and $H_2$ are bipartite. 

\textbf{Complete Bipartite $K_{a,b}$:}
\begin{itemize}
\item $|L| = a$, $|R| = b$
\item Every vertex in $L$ connected to every vertex in $R$
\item $|E| = ab$
\item Left vertices: degree $b$; Right vertices: degree $a$
\end{itemize}

\subsection{Walks}

\textbf{Walk:} Sequence $v_1, v_2, \ldots, v_n$ where $(v_i, v_{i+1}) \in E$.
\begin{itemize}
\item Can repeat vertices and edges
\item Length = number of edges traversed
\end{itemize}

\textbf{Trail:} Walk with no repeated edges (can repeat vertices).
\textbf{Path:} Walk with no repeated vertices (or edges).
\textbf{Circuit:} Trail with same start and end vertex.
\textbf{Cycle:} Path with same start and end vertex (except endpoints).
\textbf{Hierarchy:} Cycle $\subset$ Path $\subset$ Trail $\subset$ Walk

\section{Ch 2: Connectivity \& Special Walks}

\textbf{Reachability:} $u$ is reachable from $v$ if there exists a walk from $v$ to $u$.

\textbf{Connected:} Graph where every pair of vertices is reachable.

\textbf{Connected Component:} Maximal connected subgraph.

\textbf{Theorem:} Reachability is an equivalence relation:
\begin{enumerate}
\item Reflexive: $v$ reachable from $v$
\item Symmetric: $u$ from $v$ $\implies$ $v$ from $u$
\item Transitive: $u$ from $v$, $v$ from $w$ $\implies$ $u$ from $w$
\end{enumerate}

\textbf{Bridge:} Edge whose removal increases number of connected components.

\textbf{Cut Vertex:} Vertex whose removal disconnects graph.

\subsection{Bipartite Characterization}

\textbf{Lemma (Odd Loop):} If graph has odd-length loop, it has odd-length cycle.

\textbf{Theorem (Bipartite $\iff$ No Odd Cycles):} Graph $G$ is bipartite iff $G$ contains no odd-length cycle.

\textit{Proof $(\implies)$:} In cycle $v_1, v_2, \ldots, v_n, v_1$, colors alternate. If $n$ odd, $v_n$ and $v_1$ same color, contradiction.

\textit{Proof $(\impliedby)$:} Pick vertex $v$. Color vertices based on parity of distance from $v$:
\begin{itemize}
    \item Even distance $\to$ white
    \item Odd distance $\to$ black
\end{itemize}
All paths from $v$ to $w$ have same parity (else odd cycle exists). Adjacent vertices have different parity, so different colors.

\textbf{Corollary:} Every tree is bipartite (no cycles at all).

\textbf{Eulerian Circuit:} Circuit visiting each edge exactly once.
\textbf{Eulerian Graph:} Contains an Eulerian circuit.
\textbf{Eulerian Trail:} Trail visiting each edge exactly once (different start/end).

\textbf{Semi-Eulerian:} Contains an Eulerian trail.

Note: Eulerian $\implies$ Semi-Eulerian

\textbf{Theorem (Eulerian Characterization):} Finite graph $G$ is Eulerian iff:
\begin{enumerate}
    \item $G$ connected (ignoring isolated vertices)
    \item All vertices have even degree
\end{enumerate}

\textit{Proof $(\implies)$:} In circuit, entering vertex requires exiting, so degree is even.

\textit{Proof $(\impliedby)$:} Key claims:
\begin{itemize}
    \item \textbf{Claim A:} If all degrees even, edges can be partitioned into cycles
    \item \textbf{Claim B:} Cycles sharing a vertex can be merged
    \item \textbf{Claim C:} Connected graph with edge partition into cycles is Eulerian
\end{itemize}

\textbf{Theorem (Semi-Eulerian):} Finite $G$ is Semi-Eulerian iff:
\begin{enumerate}
    \item $G$ connected (ignoring isolated vertices)
    \item At most 2 vertices have odd degree
\end{enumerate}

\textit{Proof Idea:} If 2 odd-degree vertices $u, v$ exist, add edge $(u,v)$ to make all degrees even. Graph now Eulerian. Remove $(u,v)$ from circuit to get Eulerian trail.

\textbf{Constructive Algorithm:} Always choose non-bridge edge (unless no choice). Can prove no vertex surrounded only by bridges when conditions met.

\textbf{Hamiltonian Cycle:} Cycle visiting each vertex exactly once.

\textbf{Hamiltonian Path:} Path visiting each vertex exactly once.

\textbf{Hamiltonian Graph:} Contains a Hamiltonian cycle.

\textbf{WARNING:} No simple characterization like Eulerian! This is NP-complete.

\textbf{Necessary Condition:} If Hamiltonian, then connected.

\textbf{Lemma (Minimum Degree $\to$ Connected):} If $\delta(G) \geq \frac{n-1}{2}$, then $G$ is connected.

\textit{Proof:} For any $u, v$: either edge $(u,v)$ exists, or they share common neighbor. If neighborhoods disjoint, total vertices $\geq n-1$ from just two vertices, contradiction.

\textbf{Lemma (Path Extension):} If $\delta(G) \geq n/2$ and path $P$ has length $k$:
\begin{itemize}
    \item If $k < n$: can extend path or form cycle
    \item If $k = n$: have Hamiltonian path
\end{itemize}

\textit{Proof Sketch:} For maximal path $v_1, \ldots, v_m$, endpoints only connect within path. Let $S = \{i : (v_i, v_m) \in E\}$ and $T = \{i : (v_1, v_{i+1}) \in E\}$. Both $|S|, |T| \geq n/2$. If $S \cap T = \varnothing$, then $|S \cup T| \geq n$ but $|S \cup T| \leq m-1 < n$, contradiction. So overlap exists, giving cycle.

\textbf{Theorem:} If $G$ has $n > 2$ vertices and $\delta(G) \geq n/2$, then $G$ is Hamiltonian.

\textit{Proof:} Assume not Hamiltonian. Take maximal path $P$ of length $k < n$. By lemma, $P$ becomes cycle $C$ of length $k$. Since connected, some vertex $u \notin C$ connects to $C$, extending $P$. Contradiction.

\textbf{Example (Bound is Tight):} $K_{n/2} \sqcup K_{n/2}$ has $\delta = n/2 - 1$ and is disconnected (not Hamiltonian).

\section{Ch 3: Trees \& MST}
\textbf{Tree:} Connected graph with no cycles.

\textbf{Forest:} Graph with no cycles (each component is tree).

\textbf{Leaf:} Vertex of degree 1.

\textbf{Theorem (Unique Paths):} In tree, unique path exists between any two vertices.

\textit{Proof:} Connected $\implies$ path exists. If two different paths $P_1, P_2$ exist from $u$ to $v$, they diverge and reconverge, forming cycle. Contradiction.

\textbf{Theorem (Tree Edge Count):} For tree $T = (V, E)$:
$$|E| = |V| - 1$$

\textit{General Form:} For any graph with $c(G)$ connected components:
$$c(G) = |V| - |E|$$
holds iff $G$ has no cycles.

\textit{Relaxed Form:} For any graph $G$: $c(G) \geq |V| - |E|$

\textit{Proof:} Induction on $|E|$. Base: $|E| = 0 \implies c(G) = |V|$.
Step: Adding edge either:
\begin{itemize}
\item Connects two components: $c(G) \to c(G)-1$ (no cycle)
\item Stays in one component: $c(G)$ unchanged (creates cycle)
\end{itemize}

\textbf{Theorem (Minimum Leaves):} Tree with $n > 1$ vertices has at least 2 leaves.

\textit{Proof:} By Handshake Lemma:
$$\sum_{v \in V} (2 - \deg(v)) = 2n - 2(n-1) = 2$$
Each leaf contributes 1, non-leaves contribute $\leq 0$. So at least 2 leaves.

\textbf{Spanning Tree:} Subgraph that is tree with same vertex set as $G$.

\textbf{Theorem:} Every connected graph has a spanning tree.

\textit{Construction:} Start with no edges. While disconnected, add edge connecting different components.

\textbf{Depth-First Search (DFS):}
\begin{itemize}
    \item Explore as far as possible before backtracking
    \item Applications: Detect cycles, find connected components
\end{itemize}

\textbf{Breadth-First Search (BFS):}
\begin{itemize}
    \item Explore level by level
    \item Checks neighbors in order of discovery
    \item Applications: Shortest paths in unweighted graphs
\end{itemize}

\textbf{Weighted Graph:} Graph with weight function $w: E \to \R$.

\textbf{Minimum Spanning Tree (MST):} Spanning tree $T$ minimizing:
$$\sum_{e \in E(T)} w(e)$$

\textbf{Lemma (Lightest Edge in MST):} If $e$ is minimum weight edge in connected $G$, then some MST contains $e$.

If $e$ is \textit{unique} minimum, then \textit{all} MSTs contain $e$.

\textit{Proof:} Given MST $T$ without $e = (u,v)$:
\begin{itemize}
    \item Adding $e$ to $T$ creates cycle $C$
    \item $C$ contains edge $e' \neq e$ with $w(e') \geq w(e)$
    \item $T' = T \cup \{e\} \setminus \{e'\}$ is spanning tree with $w(T') \leq w(T)$
\end{itemize}

\textbf{Kruskal's Algorithm:}
\begin{enumerate}
    \item Sort edges by weight
    \item Start with $T$ = same vertices, no edges
    \item For each edge in sorted order:
    \item[] Add to $T$ if doesn't create cycle
    \item Stop when $T$ has $n-1$ edges
\end{enumerate}

\textbf{Prim's Algorithm:}
\begin{enumerate}
    \item Start with arbitrary vertex $v$, set $T = \{v\}$
    \item While $T$ doesn't span all vertices:
    \item[] Add lightest edge connecting $T$ to vertex not in $T$
\end{enumerate}

\textbf{Graph Contraction:} After adding edge $(u,v)$ to MST, can merge $u$ and $v$ into single vertex for subsequent steps.

\textbf{Cayley's Theorem:} Number of labeled trees on $n$ vertices is:
$$n^{n-2}$$

\textbf{Bijection Proof:}

Trees on $n$ vertices $\longleftrightarrow$ Lists of $n-2$ numbers from $\{1, \ldots, n\}$

\textbf{Tree $\to$ List (Prüfer Code):}
\begin{enumerate}
\item Repeatedly:
\item[] Remove smallest-index leaf
\item[] Add its neighbor to list
\item Continue until 2 vertices remain
\end{enumerate}

\textbf{List $\to$ Tree:}
\begin{enumerate}
\item Let $L$ = list, $V = \{1, \ldots, n\}$
\item Repeatedly:
\item[] Find smallest $v \in V$ not in $L$
\item[] Connect $v$ to first element of $L$
\item[] Remove $v$ from $V$, remove first element from $L$
\item When $L$ empty, connect remaining elements of $V$
\end{enumerate}

\textbf{Key Lemma:} Leaves of tree = vertices not appearing in Prüfer code.

\textit{Proof:} Non-leaf of degree $d$ appears $d-1$ times. Leaf (degree 1) appears $1-1 = 0$ times.

\textbf{Corollary:} $K_n$ has $n^{n-2}$ spanning trees.

\section{Ch 4: Structure of Connected Graphs}

\textbf{Cut Vertex:} Vertex whose removal disconnects graph.

\textbf{Bridge:} Edge whose removal disconnects graph.

\textbf{Block:} Maximal connected subgraph with no cut vertices.

\textbf{2-Connected:} Graph with no cut vertices.

\textbf{Properties:}
\begin{itemize}
\item Every bridge is incident to cut vertex (unless graph is $K_2$)
\item Trees: every edge is bridge, every non-leaf is cut vertex
\item Cycle: no bridges, no cut vertices
\end{itemize}

\section{Key Formulas \& Quick Reference}

\subsection{Edge Counts}

\begin{tabular}{ll}
$K_n$ & $|E| = \frac{n(n-1)}{2}$ \\
$K_{a,b}$ & $|E| = ab$ \\
$C_n$ & $|E| = n$ \\
$P_n$ & $|E| = n-1$ \\
Tree & $|E| = |V| - 1$ \\
Forest ($c$ comp) & $|E| = |V| - c$ \\
\end{tabular}

\subsection{Degree Bounds}

\begin{tabular}{ll}
Handshake & $\sum \deg(v) = 2|E|$ \\
Odd vertices & Even count \\
$K_n$ & $(n-1)$-regular \\
$C_n$ & 2-regular \\
\end{tabular}

\subsection{Sufficient Conditions}

\begin{tabular}{ll}
Connected & $\delta(G) \geq \frac{n-1}{2}$ \\
Hamiltonian & $\delta(G) \geq \frac{n}{2}$ \\
Eulerian & Connected + all even deg \\
Semi-Eulerian & Connected + $\leq 2$ odd deg \\
\end{tabular}

\subsection{Counting}

\begin{tabular}{ll}
Labeled trees & $n^{n-2}$ \\
Spanning trees of $K_n$ & $n^{n-2}$ \\
\end{tabular}

\section{Problem-Solving Strategies}

\subsection{Check if Eulerian}
\begin{enumerate}
\item Count odd-degree vertices
\item If 0: Eulerian
\item If 2: Semi-Eulerian
\item If $>2$: Neither
\item Check connectivity
\end{enumerate}

\subsection{Check if Hamiltonian}
No easy test! Try:
\begin{itemize}
\item $\delta(G) \geq n/2$? (sufficient)
\item Connected? (necessary)
\item Bipartite with $|L| = |R|$? (necessary if bipartite)
\item Has cut vertex? (usually not Hamiltonian)
\end{itemize}

\subsection{Find MST}
\textbf{Kruskal:} Sort edges, add if no cycle
\textbf{Prim:} Grow tree from vertex

\subsection{Count Trees}
\begin{itemize}
\item Labeled trees on $n$: use $n^{n-2}$
\item Spanning trees: Matrix-Tree Theorem
\item Unlabeled: enumerate by hand (small $n$)
\end{itemize}

\subsection{Prove by Induction}
\textbf{On $|E|$:}
\begin{enumerate}
\item Base: $|E| = 0$
\item Add/remove edge, apply IH
\item Consider cases: cycle created or not
\end{enumerate}

\textbf{On $|V|$:}
\begin{enumerate}
\item Base: $|V| = 1$ or $2$
\item Remove vertex (often leaf)
\item Apply IH, add vertex back
\end{enumerate}

\section{Common Theorems \& Techniques}

\subsection{Extremal Arguments}

\textbf{Longest Path:} Take maximal path $P$. Endpoints have special properties (all neighbors in $P$).

\textbf{Minimum Degree:} Vertex $v$ with $\deg(v) = \delta(G)$ has all neighbors with $\deg \geq \delta(G)$.

\subsection{Pigeonhole Principle}

If $|S| + |T| > |U|$ and $S, T \subseteq U$, then $S \cap T \neq \varnothing$.

Used in: Hamiltonian proof, bipartite proof.

\subsection{Path/Cycle Relationships}

\textbf{Walk $\to$ Path:} Remove loops to get path.

\textbf{Odd Loop $\to$ Odd Cycle:} Remove internal repetitions.

\textbf{Path in Tree:} Unique path between any two vertices.

\section{Worked Examples}

\subsection{Ex 1: Edge Count}

\textbf{Q:} 5-regular graph with 8 vertices. How many edges?

\textbf{A:} $\sum \deg(v) = 8 \times 5 = 40 = 2|E| \implies |E| = 20$

\subsection{Ex 2: Is Eulerian?}

\textbf{Q:} Graph with degree sequence $[2, 2, 3, 3, 4]$. Eulerian?

\textbf{A:} Two odd-degree vertices. Semi-Eulerian (if connected), not Eulerian.

\subsection{Ex 3: Tree Edges}

\textbf{Q:} Forest with 12 vertices, 3 components. How many edges?

\textbf{A:} $|E| = |V| - c = 12 - 3 = 9$ edges.

\subsection{Ex 4: Cayley's Algorithm}

\textbf{Q:} Tree with edges $(1,2), (2,3), (3,4), (3,5)$. Find Prüfer code.

\textbf{A:} Leaves: $\{1, 4, 5\}$
\begin{itemize}
\item Remove 1, add neighbor 2: $[2]$
\item Remove 4, add neighbor 3: $[2, 3]$
\item Remove 5, add neighbor 3: $[2, 3, 3]$
\end{itemize}
Code: $[2, 3, 3]$

\subsection{Ex 5: MST Weight}

\textbf{Q:} Edges: $(A,B,5), (B,C,2), (C,A,3), (A,D,1), (C,D,4)$

\textbf{A:} Kruskal (sort: 1, 2, 3, 4, 5):
\begin{itemize}
\item Add $(A,D,1)$
\item Add $(B,C,2)$
\item Add $(C,A,3)$ [connects all]
\end{itemize}
Total: $1 + 2 + 3 = 6$

\subsection{Ex 6: Hamiltonian?}

\textbf{Q:} Is $K_{3,3}$ Hamiltonian?

\textbf{A:} Yes! $|L| = |R| = 3$. Cycle exists alternating sides:
$$L_1 \to R_1 \to L_2 \to R_2 \to L_3 \to R_3 \to L_1$$

\section{Important Properties by Graph Type}

\subsection{Complete Graphs $K_n$}

\begin{itemize}
\item Eulerian: iff $n$ odd
\item Hamiltonian: always (for $n \geq 3$)
\item Not bipartite (for $n \geq 3$)
\item $n^{n-2}$ spanning trees
\end{itemize}

\subsection{Cycles $C_n$}

\begin{itemize}
\item Eulerian: always
\item Hamiltonian: always
\item Bipartite: iff $n$ even
\item No bridges, no cut vertices
\end{itemize}

\subsection{Complete Bipartite $K_{m,n}$}

\begin{itemize}
\item Eulerian: iff both $m, n$ odd
\item Hamiltonian: iff $m = n$ (both $\geq 2$)
\item Always bipartite
\item $K_{3,3}$: non-planar
\end{itemize}

\subsection{Trees}

\begin{itemize}
\item Not Eulerian (has leaves)
\item Not Hamiltonian (unless $P_n$)
\item Always bipartite
\item Every edge is bridge
\item $|E| = |V| - 1$
\end{itemize}

\subsection{Paths $P_n$}

\begin{itemize}
\item Semi-Eulerian: always (2 odd vertices)
\item Hamiltonian: by definition
\item Bipartite: always
\item Tree with exactly 2 leaves
\end{itemize}

\section{Common Mistakes}

\begin{itemize}
\item \textbf{Eulerian vs Hamiltonian:} Edges vs vertices!
\item \textbf{Trail vs Path:} Trail can repeat vertices
\item \textbf{Circuit vs Cycle:} Circuit can repeat vertices
\item \textbf{Subgraph vs Induced:} Induced has ALL edges
\item \textbf{Tree edges:} $|V| - 1$, not $|V|$
\item \textbf{Cayley:} $n^{n-2}$, not $n^n$ or $(n-1)!$
\item \textbf{Bipartite:} Triangle ($K_3$) is NOT bipartite
\item \textbf{Handshake:} Odd vertices come in pairs
\end{itemize}

\section{Proof Writing Tips}

\begin{enumerate}
\item \textbf{State assumptions:} "Let $G = (V,E)$ be a connected graph..."
\item \textbf{State goal:} "We will show that..."
\item \textbf{Use theorems:} "By Handshake Lemma..."
\item \textbf{Draw pictures:} For small cases
\item \textbf{Check edge cases:} $|V| = 1, 2$
\item \textbf{Both directions:} For "iff" proofs
\item \textbf{End clearly:} "Therefore..." or "QED"
\end{enumerate}

\section{Notation Reference}

\begin{tabular}{ll}
$G = (V, E)$ & Graph \\
$\deg(v)$ & Degree of $v$ \\
$\delta(G)$ & Min degree \\
$\Delta(G)$ & Max degree \\
$N(v)$ & Open neighborhood \\
$\ovl{N(v)}$ & Closed neighborhood \\
$K_n$ & Complete graph \\
$K_{m,n}$ & Complete bipartite \\
$C_n$ & Cycle \\
$P_n$ & Path \\
$c(G)$ & Connected components \\
$w(e)$ & Weight of edge \\
\end{tabular}

\vfill
\hrule
\vspace{1mm}
\begin{center}
\textit{Good luck on your midterm! Study the theorems and practice proofs.}
\end{center}

\end{multicols}
\end{document}
